% !TeX root = ../main.tex
\chapter{引言}
\label{cha:intro}

\section{QEMU简介}
    QEMU是一个主机上的VMM(virtual machine monitor),通过动态二进制转换
    来模拟CPU,并提供一系列的硬件模型,使guest os认为自己和硬件直接打交道,其
    实是同QEMU模拟出来的硬件打交道,QEMU再将这些指令翻译给真正硬件进行操作。

\subsection{QEMU指令翻译过程}
    QEMU模拟器的指令翻译过程采用TCG(微型代码生成器), TCG最初是C编译器的通用后端。
    它被简化为可以在QEMU中使用。指令翻译过程主要分为以下两步: 1) 将客户机程序的
    机器代码反汇编为独立于客户机架构的, 由一系列TCG ops组成的中间代码; 2) TCG
    吸收中间代码中的TCG ops, 并对它们进行一些优化, 包括活动性分析(liveness
    analysis)和琐碎的常量表达式评估(trivial constant expression evaluation).
    这些优化用于为客户机虚拟寄存器的分配提供信息, 并尽可能消去冗余的指令, 进行常量传播
    以减少简化内存操作等, 此后TCG ops中的各临时变量被分配给宿主机上的虚拟寄存器结构.
     得到最终的TCG代码并存储起来.指令的翻译以TB(Translation Block)为单位, 单个TB
     的大小取决于两点: 1) 每个TB内最多能声明512个临时变量; 2)每个TB最多能有两个跳转
     出口。最终生成的TCG代码描述了一系列客户机虚拟寄存器及内存之间的操作。

\subsection{QEMU指令执行过程}
    QEMU模拟器的指令翻译过程采用TCI(微型代码解释器), TCI模拟了一个虚拟的客户机CPU,
    对指令翻译过程生成的TCG代码逐条解释执行, 执行过程中会模拟客户机CPU,进行一系列虚拟
    寄存器和内存之间的操作. 至于各种逻辑与算术运算, TCI直接用C语言的相对应操作进行模拟。

\subsection{TB chaining}
    QEMU模拟器在指令翻译得到TB后, 记录下该TB的第一条指令的guest pc与该TB的对应关系, 并
    存储在多级缓存中。在执行一个TB结束后, QEMU使用模拟的程序计数器(PC)和其他的虚拟CPU状态信息,
    在哈希表中查找下一个TB的地址, 若哈希表中查找失败, 进而会在多级缓存中进行查找~\cite{qemuBellard05}。这一查找过程
    带来的性能开销较大。而TB chaining机制有效降低了TB查找的开销。TB chaining是指在某一次
    执行TB后, 记录下该次跳转出口所到达的目的TB的TCG code地址, 这样使得下一次执行至该TB,
    并从同一跳转出口离开该TB时, 无须再根据pc查询下一TB的地址, 而可以直接跳转至下一TB的TCG code处开始执行。
    TB chaining使得在不考虑TB失效的前提下, 对于每个TB跳转, 只需要进行+一次TB查找, 大大减少了在缓存中查询TB的次数.


\section{LLVM简介}
    LLVM是一个针对多种体系结构的编译器框架, 是一个用于构建编译器的工具包,是将指令转换为
    计算机可以读取和执行的形式的程序。它會將中間語言(Intermediate Representation
    ,IR)从编译器取出并最佳化,最佳化后的IR接着被转换及链接到目标平台的汇编语言。LLVM可以接
    受來自GCC工具链所编译的IR,包含它底下现存的编译器。

\subsection{LLVM IR}
    LLVM的一个优势就是提供了统一的内部表示(LLVM IR), LLVM IR旨在为编译器的优化器部分提供基
    础, LLVM IR是一种抽象的低级指令, 是一个像RISC的指令集, 然而可以很好地表达更为高级的信息,
    或者说高级语言可以很好地映射到LLVM IR。LLVM IR使得编译器的各个前端(高级语言)和多个后端(体
    系结构)能够共享中间表示, 共享代码优化器。

\subsection{LLVM JIT}
    LLVM与静态编译不同, 它可以在编译时期, 链接时期, 甚至是执行时期产生可重新定位的机器码。LLVM
    JIT(Just-in-time) 编译器的目的是在需要时"即时"编译代码。具体的说, 程序被人为划分为多个
    编译模块, 当这个模块中的指令第一次需要被执行时, 才编译整个模块, 即对该模块的LLVM IR执行一
    系列优化并产生机器码。采用JIT编译的系统通常会持续分析正在执行的代码, 并识别出从重新编译中获得
    的执行性能提升能超过编译开销的部分代码。JIT编译将静态编译的执行效率与解释器的灵活性结合在一起,
    同时也结合了两者的额外开销。由于加载和编译字节码所花费的时间,JIT在应用程序的初始执行中引起了
    轻微到明显的延迟。 有时,这种延迟称为“启动时间延迟”或“预热时间”。 通常,JIT执行的优化越多,它
    将生成的代码越好,但是初始延迟也会增加。 因此,JIT编译器必须在编译时间和希望生成的代码质量之间
    进行权衡。

\section{使用LLVM改善QEMU指令翻译过程的好处}
    当前QEMU使用翻译得到的TCG code作为中间代码, 后端用软件模拟一个虚拟的客户机CPU, 并由TCI解释
    执行TCG code. 我的实现采用LLVM IR代替TCG code作为中间代码, 执行指令时有两种选择:1)对IR
    进行一系列优化编译得到机器代码, 直接在宿主机上运行; 2)直接用LLVM IR解释器解释执行; 用LLVM改
    善QEMU的指令翻译过程, 会带来以下几点好处:

\subsection{更加丰富的IR优化类型}
    当前的QEMU模拟器在得到TCG中间代码后, 仅仅进行了以下几项优化: 活动性分析, 常量传播, 冗余指令/变量
    消除等, 然后将变量分配给虚拟CPU中的虚拟寄存器。而使用LLVM IR作中间代码, 可以执行种类更加繁多的
    优化, 能更大程度的减小最终生成代码的体积, 获取更佳的执行性能。

\subsection{针对不同的体系结构生成机器代码}
    由于每个TB实际上只会包含最多几百条指令, LLVM提供的种类繁多的复杂的IR优化, 相较于TCG优化而言, 实
    际上能额外提供的优化限度相对有限。使用LLVM进行翻译过程优化的真正优势在于, LLVM的代码生成器能根据
    宿主机的体系结构, 将LLVM IR编译生成最适合与在宿主机上运行的机器码。

    当前QEMU模拟器使用一个虚拟宿主机CPU逐条解释执行TCG code中的指令, 用软件模拟真实CPU带来的性能开销
    是巨大的。例如, QEMU在指令执行过程对虚拟寄存器的访问实际上是对栈上的内存空间的访问, 这与真实的硬件寄
    存器的访问速度有一定差距。另外, TCG code的生成是独立于宿主机架构的, 即TCG无法得知宿主机的架构, 同
    一个客户机程序在不同的宿主机架构上都会得到相同的TCG code,虚拟CPU相当于运行在宿主机上的一个用户程序,
    这样无法充分利用不同体系结构的特点, 充分利用架构特点得到适合的代码。而使用LLVM的代码生成器可以根据特定宿
    主机的架构(内存布局, 指令集, 寄存器等), 将LLVM IR编译为适合于在宿主机运行的机器码, 并直接由宿主机CPU
    执行,充分利用硬件优势。这与QEMU模拟器的后端逐条解释执行独立于宿主机体系结构的TCG code相比, 性能上有
    着巨大优势。

\subsection{即时编译}
    LLVM的即时编译机制与QEMU原本的动态翻译思想可以很好的契合, TCG翻译的单位-——TB可以作为JIT的编译模块,
    JIT在需要时执行一个TB的编译。

    综上所述, 我们有理由相信, 使用LLVM对QEMU模拟器进行优化, 即使用LLVM IR代替TCG code作为中间代码,
    并用LLVM JIT代替虚拟CPU进行后端机器码的执行, 会使得QEMU模拟器的性能得到提升。



